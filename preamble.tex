%%%%%%%%%%%%%%%%%%%%%%%%%%%%%%%%%%%%%%%%%%%%%%%%%%%%%%%%%%
% Most students will not need to edit this file.
% Only edit if you are sure you know what you are doing.
%%%%%%%%%%%%%%%%%%%%%%%%%%%%%%%%%%%%%%%%%%%%%%%%%%%%%%%%%%
\usepackage[sort]{natbib}
\usepackage{url}
\usepackage{graphicx}
\usepackage{booktabs}
\usepackage{longtable}
\usepackage{array}
\usepackage{multirow}
\usepackage{wrapfig}
\usepackage{float}
\usepackage{colortbl}
\usepackage{pdflscape}
\usepackage{tabu}
\usepackage{threeparttable}
\usepackage{threeparttablex}
\usepackage[normalem]{ulem}
\usepackage{makecell}
\usepackage{xcolor}
\usepackage{parskip}
\usepackage{fancyhdr}
\usepackage[T1]{fontenc}
\usepackage{verbatim}
\usepackage{setspace}
\usepackage{mathtools}
\usepackage{amssymb}
\usepackage[left=37mm,right=30mm,top=35mm,bottom=30mm]{geometry}
\usepackage[amsmath,thmmarks]{ntheorem}
\usepackage{todonotes}


\renewcommand{\bibname}{References}

\setlength{\theorempreskipamount}{3.0ex plus 1ex minus 0.75ex}
\setlength{\theorempostskipamount}{3.0ex plus 1ex minus 0.75ex}

\theorembodyfont{\normalfont} \theoremstyle{plain}
\newtheorem{theorem}{Theorem}[section]
\newtheorem{exa}{Example}[section]
\newtheorem{corollary}[theorem]{Corollary}
\newtheorem{lemma}[theorem]{Lemma}
\newtheorem{proposition}[theorem]{Proposition}


\newtheorem{definition}[theorem]{Definition}
\newtheorem{remark}[theorem]{Remark}
\newtheorem{notation}[theorem]{Notation}
\newtheorem{assumption}[theorem]{Assumption}
\newtheorem{conjecture}[theorem]{Conjecture}

\setlength{\headheight}{16pt}


\frontmatter
